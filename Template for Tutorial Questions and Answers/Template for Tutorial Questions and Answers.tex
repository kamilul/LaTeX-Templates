
%%%%%%%%%%%%%%%%%%%%%%%%%%%%%%%%%%%%%%%%%%%%%%%%%%%%%%%%%%%%%%%%%%%%%%%
% 
% LaTeX Template -----------------------------------------------------
%       Name: Template for Tutorial Questions & Answers
%       Description: The template is used for displaying tutorial
%       questions with the option to show questions only, or both
%       questions and answers
%
% Author Details -----------------------------------------------------
%       Kamilul Ashraf Bin Kamsani
%       National University of Singapore
%       kamilul@comp.nus.edu.sg
%
% Documentation ------------------------------------------------------
%
%       * At line 29, you can choose either
%   (i)  Show ques only: \documentclass[11pt,a4paper]{exam} or 
%   (ii) Show ques and ans: \documentclass[11pt,a4paper,answers]{exam} 
%
%       * Compile the .tex document to the see the difference
%   (i)  For show ques, document name will change to "Questions", the 
%        author's name and the disclaimer message will not appear
%   (ii) For show ques and ans, document name will change to "Answers",
%        the author's name and the disclaimer message will appear
% 
%%%%%%%%%%%%%%%%%%%%%%%%%%%%%%%%%%%%%%%%%%%%%%%%%%%%%%%%%%%%%%%%%%%%%%%%

\documentclass[10pt,a4paper,answers]{exam}

%% - Required Packages
\RequirePackage{amssymb, amsfonts, amsmath, latexsym, verbatim, xspace, setspace}
\RequirePackage{tikz, pgflibraryplotmarks}

%% - Commands
\renewcommand{\solutiontitle}{\noindent\textbf{Answers:}\par\noindent} % default word is "Solutions:"
\newcommand{\choosedocumenttitle}{\ifprintanswers\documenttitle \\ \else \altdocumenttitle \\ \fi} % doc. title will change, ie. answers or questions
\newcommand{\chooseshowname}{\ifprintanswers \authorname \\ \authorschool \\ \em \authoremail \fi} % author info will not show when in questions mode
\newcommand{\choosedate}{\ifprintanswers \vspace{-3ex} \else \vspace{-9ex} \fi}

%% - Document Information
\newcommand{\documenttitle}{Answers for Tutorial \#2} % doc. title for questions and answers
\newcommand{\altdocumenttitle}{Questions for Tutorial \#2} % alt. doc. title for questions only
\newcommand{\modulename}{IS2102: Requirements Analysis and Design} % module name
\newcommand{\authorname}{Kamilul Ashraf Kamsani} % author name
\newcommand{\authorschool}{National University of Singapore} % author's institution
\newcommand{\authoremail}{kamilul@comp.nus.edu.sg} % author's email

%% - Title
\title{\choosedocumenttitle \vspace{-1ex}\begin{large}\em \modulename \end{large} \\}
\author{\chooseshowname}
\date{\choosedate}

%----------------------------------------------------------------------
%       DOCUMENT CONTENT
%----------------------------------------------------------------------

\begin{document}
\maketitle
\thispagestyle{empty} % enable first page to NOT show page number
\begin{questions}

%----------------------------------------------------------------------
%       QUESTION 1
%----------------------------------------------------------------------

\question What are the phases in most SDLCs?
\begin{solution}
\begin{itemize}

        \item Initiation during which a business case is made for the project (includes Business Modeling discipline)
        \item Discovery during which the eliciting, analysis, and documentation of detailed requirements peaks.
        \item Construction during which the IT solution (or Information System) is built (includes Design, Implementation, Testing disciplines)
        \item Final Verification and Validation (includes Testing discipline)
        \item Closeout (includes Deployment discipline)
        
\end{itemize}
\end{solution}

%----------------------------------------------------------------------
%       QUESTION 2
%----------------------------------------------------------------------

\question What are the objectives of the initiation phase?
\begin{solution}
\begin{itemize}

        \item To develop the business case for the project
        \item To establish project and product scope
        \item To explore solutions, including the preliminary architecture

\end{itemize}
\end{solution}

%----------------------------------------------------------------------
%       QUESTION 3
%----------------------------------------------------------------------

\question What are the deliverables of the initiation phase?
\begin{solution}
\begin{itemize}

        \item A single document: Business Requirements Document (BRD) to describe business requirements
        \item The BRD will be revised as the project progresses
        \item Key components of the BRD produced during the Initiation phase include

        \begin{itemize}
        \item Business use-case descriptions include business use-case diagrams
        \item Role map
        \item System use-case diagram
        \item Initial class diagram describing key business classes
        \end{itemize}

\end{itemize}
\end{solution}

%----------------------------------------------------------------------
%       QUESTION 4
%----------------------------------------------------------------------

\question What are the elements of an activity diagram?
\begin{solution}
\begin{itemize}

        \item \textbf{Initial node:}  indicates where the workflow begins
        \item \textbf{Activity:} indicates a step in the process. Notice anything about the typical naming convention?
        \item \textbf{Control flow:} an arrow showing the direction of the workflow
        \item \textbf{Decision:} a diamond symbol, indicating a possibility of different paths
        \item \textbf{Guard condition:} a condition attached to a control flow. A guard is shown within square brackets
        \item \textbf{Merge:} model a number of alternative flows that lead to the same activity
        \item \textbf{Event:} a trigger attached to a control flow and is indicated without the use of square brackets
        \item \textbf{Final node:} indicates the end of the process
        \item \textbf{Fork and Join:} bars used to document parallel activities.
        \begin{itemize}
                \item A \textbf{fork} indicates the point after which a number of activities may begin in any order. 
                \item A \textbf{join} indicates that workflow may commence only once the parallel activities that flow into it have all been completed.
        \end{itemize}

\end{itemize}
\end{solution}

%----------------------------------------------------------------------
%       QUESTION 5
%----------------------------------------------------------------------

\question What are some of the modeling elements in business use-case diagram?
\begin{solution}
\begin{itemize}

        \item \textbf{Business actor:} someone external to the business
        \item \textbf{Worker:} someone who works within the business
        \item \textbf{Association:} indicates that the actor interacts with the business over the course of the business use case

\end{itemize}
\end{solution}

%----------------------------------------------------------------------
%       QUESTION 6
%----------------------------------------------------------------------

\question What is the purpose of an activity diagram and what is the difference between an activity and an action?
\begin{solution}
\begin{itemize}

        \item An activity diagram models the work flow of each business use case
        \item An action is a simple, nondecomposable piece of behavior, while an activity is used to represent a set of actors.

\end{itemize}
\end{solution}

%----------------------------------------------------------------------
%       QUESTION 7
%----------------------------------------------------------------------

\question What is meant by polymorphism when applied to object oriented systems?
\begin{solution}

        The polymorphic methods in the classes can have many different implementations according to number of parameters or the subclasses implementing the abstract methods.

\end{solution}

%----------------------------------------------------------------------
%       QUESTION 8
%----------------------------------------------------------------------

\question What techniques are used by analysts when they need to guide users in explaining what is wanted from a system?
\begin{solution}
\begin{itemize}

        \item Interviews (Top-down or Bottom-up)
        \item Joint Application Development (JAD)
        \item Questionnaires

\end{itemize}
\end{solution}

%----------------------------------------------------------------------
%       DISCLAIMER
%       You can remove the entire section if not required
%%----------------------------------------------------------------------

 \ifprintanswers \begin{small}\em \underline{Disclaimer:} These answers are not necessarily correct, infact they might be wrong! (because I blindly copied it from the slides) Hence, I'd be happy if you are willing to share your answers to correct the ones above. Email me at \authoremail!\end{small} \fi

%%%%%%%%%%%%%%%%%%%%%%%%%%%%%%%%%%%%%%%%%%%%%%%%%%%%%%%%%%%%%%%%%%%%%%

\end{questions}
\end{document}